\documentclass[a4paper,12pt]{article} % добавить leqno в [] для нумерации слева

%%% Работа с русским языком
\usepackage{cmap}					% поиск в PDF
\usepackage{mathtext} 				% русские буквы в фомулах
\usepackage[T2A]{fontenc}			% кодировка
\usepackage[utf8]{inputenc}			% кодировка исходного текста
\usepackage[english,russian]{babel}	% локализация и переносы

%%% Дополнительная работа с математикой
\usepackage{amsmath,amsfonts,amssymb,amsthm,mathtools} % AMS
\usepackage{icomma} % "Умная" запятая: $0,2$ --- число, $0, 2$ --- перечисление

%% Номера формул
\mathtoolsset{showonlyrefs=true} % Показывать номера только у тех формул, на которые есть \eqref{} в тексте.

%% Шрифты
\usepackage{euscript}	 % Шрифт Евклид
\usepackage{mathrsfs} % Красивый матшрифт

%% Свои команды
\DeclareMathOperator{\sgn}{\mathop{sgn}}

%% Перенос знаков в формулах (по Львовскому)
\newcommand*{\hm}[1]{#1\nobreak\discretionary{}
{\hbox{$\mathsurround=0pt #1$}}{}}

%%% Заголовок
\author{\LaTeX{} в Вышке}
\title{1.2 Математика в \LaTeX}
\date{\today}

\begin{document} % конец преамбулы, начало документа

\maketitle

\subsection{Системы уравнений}

\[
	\left\{
		\begin{aligned}
			ax^2 + bx + c = 0  \\
			ax^2 + bx + c = 0 \\
			ax^2 + bx + c = 0\\
		\end{aligned}
	\right.
\]

\[
	|x|=\begin{cases}
		x, &\text{если }  x \ge 0 \\
		-x, &\text{если } x<0
	\end{cases}
\]

\section{Матрицы}

\[
	\begin{pmatrix}
		a_{11} & a_{12} & a_{13} \\
		a_{21} & a_{22} & a_{23}
	\end{pmatrix}
\]

\[
	\begin{vmatrix}
		a_{11} & a_{12} & a_{13} \\
		a_{21} & a_{22} & a_{23}
	\end{vmatrix}
\]

\[
	\begin{bmatrix}
		a_{11} & a_{12} & a_{13} \\
		a_{21} & a_{22} & a_{23}
	\end{bmatrix}
\]

В уравнении \eqref{eq:sum} на стр. \pageref{eq:sum} много слагаемых.

\end{document} % конец документа
